%#############################PREAMBLE#############################################
\documentclass[twoside,11pt,]{article}

\usepackage[spanish]{babel}
\usepackage{graphicx}
\usepackage{float}
\usepackage[skins]{tcolorbox}
\usepackage{titlepic}
\usepackage{fancyhdr}
\usepackage{geometry}

\geometry{a4paper, total={170mm,257mm}, left=20mm, top=25mm,}

\pagestyle{fancy}
\lhead{Cartografia Geotecnica}
\rhead{\thepage}
\cfoot{Programa}
\renewcommand{\headrulewidth}{0.4pt}
\renewcommand{\footrulewidth}{0.4pt}
\renewcommand{\labelenumi}{Clase \theenumi}

\graphicspath{{G:/My Drive/FIGURAS/}}

\title {PROGRAMA  CURSO\\ CARTOGRAFIA GEOTECNICA}
\author{Prof.: Edier Aristizábal\\[5ex]
\includegraphics[width=10.0cm]{unal2}
}
\date{}

%################################BODY############################################
\begin{document}
\maketitle

\emph {versión}: \today

\section* {Introducción}
El curso de \textbf{Cartografía Geotécnica} está orientado para estudiantes de ingeniería que deseen adquirir conocimientos sobre métodos para la evaluación y zonificación de la susceptibilidad y amenaza por movimientos en masa y avenidas torrenciales en zonas de montaña, con enfoque en ordenamiento territorial.
\par El curso es teórico - práctico, en el cual se hace una revisión detallada del estado del arte y las diferentes metodologías de zonificación utilizadas alrededor del mundo, incluyendo métodos de aprendizaje automático (\emph{machine learning}), minería de datos (\emph{data mining}), y análisis espacial de datos.
\par El alcance de este curso es aprender a construir modelos para cartografiar y construir mapas de susceptibilidad y amenaza por movimientos en masa como herramienta para el ordenamiento del territorio. Dichos métodos pueden aplicarse a otro tipo de amenazas.

\section{RECOMENDACIÓN}
Para tomar el curso se recomienda al estudiante haber realizado su núcleo básico y los cursos SIG y Sensores Remotos. Para un máximo beneficio del contenido del curso es ideal tener conocimientos en programación, que le permita el manejo de grandes volúmenes de información (\emph{big data}), como bases de datos y mapas tipo raster.

\section{PROGRAMA}
El contenido del curso comprende los siguientes temas a desarrollar:\\

\subsection*{Introducción al curso}

\subsection {Geoamenazas \& Ordenamiento territorial}

\begin{tcolorbox}[enhanced,width=5in,center upper,  fontupper=\large\bfseries,drop shadow southwest,sharp corners]
Lectura \& Quiz
\end{tcolorbox}

\subsection {Hidrología de ladera \& estabilidad}

\begin{tcolorbox}[enhanced,width=5in,center upper,  fontupper=\large\bfseries,drop shadow southwest,sharp corners]
Lectura \& Quiz
\end{tcolorbox}

\subsection {Factores condicionantes \& detonantes}

\begin{tcolorbox}[enhanced,width=5in,center upper,  fontupper=\large\bfseries,drop shadow southwest,sharp corners]
Lectura \& Quiz
\end{tcolorbox}

\subsection {Principios de zonificación}

\begin{tcolorbox}[enhanced,width=5in,center upper,  fontupper=\large\bfseries,drop shadow southwest,sharp corners]
Lectura \& Quiz
\end{tcolorbox}

\subsection {Intro análisis geoespacial}

\begin{tcolorbox}[enhanced,width=5in,center upper,  fontupper=\large\bfseries,drop shadow southwest,sharp corners]
Lectura
\end{tcolorbox}

\subsection {Inventario de eventos}

\begin{tcolorbox}[enhanced,width=5in,center upper,  fontupper=\large\bfseries,drop shadow southwest,sharp corners]
Taller 1 -- Inventario de eventos
\end{tcolorbox}

\subsection {Exploración y selección de variables}
Variables condicionantes, Variable dependiente, Variables continuas, categóricas, Indices, buffer, Histogramas, Matriz de correlación.
-	Análisis de componentes principales (PCA)

\begin{tcolorbox}[enhanced,width=5in,center upper,  fontupper=\large\bfseries,drop shadow southwest,sharp corners]
Taller 2 -- Selección de variables
\end{tcolorbox}

\subsection {MÉTODOS HEURÍSTICOS}
Cartografía geomorfológica directa, Algebra de mapas, Análisis multicriterio

\begin{tcolorbox}[enhanced,width=5in,center upper,  fontupper=\large\bfseries,drop shadow southwest,sharp corners]
Taller 3 -- Métodos heurísticos
\end{tcolorbox}

\subsection {MÉTODOS ESTADÍSTICOS BIVARIADOS}
Métodos bivariados, Likelihood, Frecuency ratio model, Evidential Belief Function, Certainty factor, Statistical index model, Information value model, Weights of evidence

\begin{tcolorbox}[enhanced,width=5in,center upper,  fontupper=\large\bfseries,drop shadow southwest,sharp corners]
Taller 4 -- Métodos bivariados
\end{tcolorbox}

\subsection {MÉTODOS ESTADÍSTICOS MULTIVARIADOS}
Métodos multivariados, Análisis condicional, Regresión logística, Análisis discriminante, Arbol de decision (tree decision) \& Random forest, Redes neuronales.

\begin{tcolorbox}[enhanced,width=5in,center upper,  fontupper=\large\bfseries,drop shadow southwest,sharp corners]
Taller 5 -- Métodos multivariados
\end{tcolorbox}

\subsection {MÉTODOS CON BASE FÍSICA}
SHALSTAB, TRIGRS, SHIA\_Landslide
Análisis determinístico -- probabilístico, Modelos acoplados
\begin{tcolorbox}[enhanced,width=5in,center upper,  fontupper=\large\bfseries,drop shadow southwest,sharp corners]
Taller 6 -- Métodos físicos acoplados
\end{tcolorbox}

\subsection {VALIDACIÓN Y CLASIFICACIÓN}
Matriz de confusión y estadísticos, Coeficiente de Kappa - Cohen, Curvas ROC y área bajo la curva, Curvas de éxito, Distancia a la clasificación perfecta, Grado de ajuste.

\begin{tcolorbox}[enhanced,width=5in,center upper,  fontupper=\large\bfseries,drop shadow southwest,sharp corners]
Taller 7 -- Evaluación de modelos
\end{tcolorbox}

\subsection {PROBABILIDAD TEMPORAL \& MAGNITUD}
Probabilidad condicionada, Regresión logística, Poisson, Binomial, Distribución área vs. frecuencia.


\subsection {MÉTODOS DE PROPAGACIÓN}
Flow – R, Autómatas celulares.

\begin{tcolorbox}[enhanced,width=5in,center upper,  fontupper=\large\bfseries,drop shadow southwest,sharp corners]
Taller 8 -- Propagación 
\end{tcolorbox}

\section{EVALUACIÓN DEL CURSO}
El curso se evaluará mediante \emph{quizes} con un valor del 5\% y talleres para la zonificación de la susceptibilidad y/o amenaza de una cuenca seleccionada con un valor del 10\%. Para esto se deberán conformar equipos de 3 personas y deberán adquirir la información necesaria de cada cuenca. La búsqueda y obtención de información adecuada hace parte de la evaluación de los Talleres. Cada grupo deberá elaborar los talleres en su cuenca de estudio.
\par Los talleres deben ser entregados en formato digital (PDF) con el número del taller y el estudiante (EJ. Taller1\_EdierAristizabal), y deberá ser remitido al correo evaristizabalg@unal.edu.co hasta las 8 a.m. del día definido en la siguiente tabla.

\begin{table}[!hbt]
\centering
\label{tab-marks}
\begin{tabular}{|l|c|}
\hline {\bf Talleres} & {\bf Fecha entrega}\\
\hline Taller 1 & 10/09/2020\\
\hline Taller 2 & 17/09/2020\\
\hline Taller 3 & 24/09/2020\\
\hline Taller 4 & 01/10/2020\\
\hline Taller 5 & 08/10/2020\\
\hline Taller 6 & 15/10/2020\\
\hline Taller 7 & 22/10/2020\\
\hline Taller 8 & 29/10/2020\\
\hline
\end{tabular}
\end{table}

\section{REFERENCIAS}
El curso utilizará material de diferentes fuentes bibliográficas, entre las cuales se destacan las siguientes, por lo cual se recomienda su consulta:
\begin{itemize}
\item Guidelines for landslide susceptibility, hazard and risk assessment and zoning. 2011. SafeLand. Technical University of Catalonia (UPC). pp. 173.
\item Guzzetti F. 2011. Landslide hazard and risk assessment. PhD  thesis. University of Bonn (Germany). 
\item Landslides, investigation and mitigation, Special report 247. TRB. 1996.
\end{itemize}

\end{document}
