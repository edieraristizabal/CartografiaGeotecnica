%Modo presentación
\documentclass[12pt]{beamer}

%Modo handout
%\documentclass[handout,compress]{beamer}
%\usepackage{pgfpages}
%\pgfpagesuselayout{4 on 1}[border shrink=1mm]

\usepackage{graphicx}
\usepackage{beamerthemeCambridgeUS}
\usepackage{subfig}
\usepackage{tikz}
\usepackage{amsmath}
\usepackage{hyperref}
\usepackage{ragged2e}

\graphicspath{{G:/My Drive/FIGURAS/}}

\title[Inventario]{CARTOGRAFÍA GEOTÉCNICA}
\author[Edier Aristizábal]{Edier V. Aristizábal G.}
\institute{\emph{evaristizabalg@unal.edu.co}}
\date{Version:\today}

\usepackage{textpos} 

\addtobeamertemplate{headline}{}{%
	\begin{textblock*}{2mm}(.9\textwidth,0cm)
	\hfill\includegraphics[height=1cm]{un}  
	\end{textblock*}
			}
%############################INICIO#############################################
\begin{document}
\begin{frame}
\titlepage
\centering
	\includegraphics[width=5cm]{unal}\hspace*{4.75cm}~%
   	\includegraphics[width=2cm]{logo3} 
\end{frame}
 %#############################SLIDE
\begin{frame}
\frametitle{Inventario de MenM}
\justifying
\scriptsize{Colección de las características de los deslizamientos en una cierta área y tiempo, preferiblemente en formato digital con información espacial relacionada con la localización (puntos o polígonos) combinada con información de atributos. Estos atributos deben idealmente contener información sobre:}
\begin{columns}
\begin{column}{0.45\linewidth}
\begin{itemize}
\item el tipo de deslizamiento, 
\item actividad actual,
\item tamaño y/o volumen, 
\item fecha de ocurrencia o edad relativa, 
\item causas. 
\end{itemize}
\end{column}
\begin{column}{0.55\linewidth}
\begin{figure}
   	\includegraphics[scale=0.3]{landslides}
\end{figure}
\end{column}
\end{columns}
\tiny{Fuente: AGS (2007), Corominas et al. (2014)}
\end{frame}
%#############################SLIDE
\begin{frame}
\frametitle{Por qué son importantes?}
\scriptsize{
\begin{itemize}
\item Los inventarios de deslizamientos son la base para evaluar la susceptibilidad, amenaza y riesgo por deslizamientos.\\
\vspace{0.3cm}
\item Son esenciales para los modelos de susceptibilidad que predicen los deslizamientos basados en las condiciones pasadas: Necesitamos saber dónde ocurren y cuántos.\\
\vspace{0.3cm}
\item Esas condiciones son utilizadas para predecir los deslizamientos futuros: Necesitamos saber los factores causantes.
\item Estas condiciones difieren para diferentes tipos de movimientos: Necesitamos saber qué pasó.\\
\vspace{0.3cm}
\item La información temporal es esencial para estimar la frecuencia de los deslizamientos: Necesitamos saber cuándo pasaron.\\
\vspace{0.3cm}
\item Los inventarios de deslizamientos son usados para validar los resultados de la zonificación.
\end{itemize}
}
\vspace{1cm}
\tiny{Fuente: van Westen en ACP-EU Natural Disaster Risk Program }
\end{frame}
%%%%%%%%%%%%%%%%%%%%%%%%%%%%%%%%%%%%%%%%%%%%%%%%%%%%%%%%%%%%%%%%%%%%%%%%%%%%%%%
\begin{frame}
\frametitle{Técnicas de levantamiento}
\scriptsize{
\begin{itemize}
\item Análisis de archivos información histórica (Reichenbach et al., 1988; Salvati et al., 2003).\\
\vspace{0.3cm}
\item Fotointerpretación de fotografías aéreas (Guzzetti \& Cardinali, 1989, 1990; Galli et al., 2008; Santagelo et al., 2010, 2013).\\
\vspace{0.3cm}
\item Análisis visual de imágenes LIDAR (Ardizzone et a., 2007; van den Eeckhaut et al., 2007; Haneberg et al., 2009; Guzzetti et al., 2012; Razak et al., 2011, 2013).\\
\item Procesamiento de imágenes LIDAR (Martha et al., 2010; Lu et al., 2011, van der Eeckhaut et al., 2012).\\
\vspace{0.3cm}
\item Procesamiento de imágenes de satélite (Mondini \& Chang, 2014; Yang \& Chen, 2010; Rosin \& Hervas, 2005).\\
\vspace{0.3cm}
\item Levantamiento de campo.\\
\end{itemize}
}
\vspace{1cm}
\tiny{Fuente: van Westen en Landslide types and causes; Santagelo et al. (2015)}
\end{frame}
%%%%%%%%%%%%%%%%%%%%%%%%%%%%%%%%%%%%%%%%%%%%%%%%%%%%%%%%%%%%%%%%%%%%%%%%%%%%%%%
\begin{frame}
\frametitle{Técnicas de levantamiento}
\begin{figure}
   	\includegraphics[scale=0.43]{inventario-tecnicas}
\end{figure}
\tiny{Fuente: Reichenbach et al. (2018)}
\end{frame}
%%%%%%%%%%%%%%%%%%%%%%%%%%%%%%%%%%%%%%%%%%%%%%%%%%%%%%%%%%%%%%%%%%%%%%%%%%%%%%%
\begin{frame}
\frametitle{Técnicas de levantamiento}
\begin{figure}
   	\includegraphics[scale=0.55]{inventario-tecnicas-tabla}
\end{figure}
\end{frame}
%%%%%%%%%%%%%%%%%%%%%%%%%%%%%%%%%%%%%%%%%%%%%%%%%%%%%%%%%%%%%%%%%%%%%%%%%%%%%%%
\begin{frame}
\frametitle{Inventario histórico}
\url{http://www.emdat.be/}
\begin{figure}
   	\includegraphics[scale=0.52]{emdat}
\end{figure}
\end{frame}
%%%%%%%%%%%%%%%%%%%%%%%%%%%%%%%%%%%%%%%%%%%%%%%%%%%%%%%%%%%%%%%%%%%%%%%%%%%%%%%
\begin{frame}
\frametitle{Inventario histórico}
\url{http://desinventar.net}
\begin{figure}
   	\includegraphics[scale=0.52]{desinventar-net}
\end{figure}
\end{frame}
%%%%%%%%%%%%%%%%%%%%%%%%%%%%%%%%%%%%%%%%%%%%%%%%%%%%%%%%%%%%%%%%%%%%%%%%%%%%%%%
\begin{frame}
\frametitle{Inventario histórico}
\scriptsize{\url{https://data.nasa.gov/dataset/Global-Landslide-Catalog-Export/dd9e-wu2v}}
\begin{figure}
   	\includegraphics[scale=0.52]{inventario-nasa}
\end{figure}
\end{frame}
%%%%%%%%%%%%%%%%%%%%%%%%%%%%%%%%%%%%%%%%%%%%%%%%%%%%%%%%%%%%%%%%%%%%%%%%%%%%%%%
\begin{frame}
\frametitle{Inventario histórico}
\url{https://desinventar.org}
\begin{figure}
   	\includegraphics[scale=0.48]{desinventar-org}
\end{figure}
\end{frame}
%%%%%%%%%%%%%%%%%%%%%%%%%%%%%%%%%%%%%%%%%%%%%%%%%%%%%%%%%%%%%%%%%%%%%%%%%%%%%%%
\begin{frame}
\frametitle{Inventario histórico}
\url{http://simma.sgc.gov.co}
\begin{figure}
   	\includegraphics[scale=0.48]{simma}
\end{figure}
\end{frame}
%%%%%%%%%%%%%%%%%%%%%%%%%%%%%%%%%%%%%%%%%%%%%%%%%%%%%%%%%%%%%%%%%%%%%%%%%%%%%%%
\begin{frame}
\frametitle{Inventario histórico}
\justifying
\scriptsize{
\textbf{Inventario de deslizamientos multi-temporal}: son comúnmente utilizados en modelos de susceptibilidad y amenaza. Entre mas largo el periodo que cubre, mayor significancia temporal del inventario.\\
\vspace{0.5cm}
\textbf{Inventario de un solo periodo}: corresponden a inventarios de deslizamientos de un cierto periodo de tiempo a partir de fotos aéreas o imágenes de satélite. No son validos para calibrar un modelo de susceptibilidad.\\
\vspace{0.5cm}
\textbf{Inventario de deslizamientos de un evento}: inventarios de deslizamientos que ocurren como resultado de un evento detonante particular (lluvia, sismo). Son adecuados para análisis de susceptibilidad basados. Su elaboración puede ser mas sencilla, especialmente con imágenes de alta resolución espacial y temporal fácilmente obtenibles en la actualidad.}\\
\vspace{2cm}
\tiny{Fuente: Lee (2014) SLdR}
\end{frame}
%%%%%%%%%%%%%%%%%%%%%%%%%%%%%%%%%%%%%%%%%%%%%%%%%%%%%%%%%%%%%%%%%%%%%%%%%%%%%%%
\begin{frame}
\frametitle{Inventario multitemporal}
\begin{figure}
   	\includegraphics[scale=0.45]{inventario-multitemporal}
\end{figure}
\tiny{Fuente: Samia et al. (2016)}
\end{frame}
%%%%%%%%%%%%%%%%%%%%%%%%%%%%%%%%%%%%%%%%%%%%%%%%%%%%%%%%%%%%%%%%%%%%%%%%%%%%%%%
\begin{frame}
\frametitle{Inventario evento}
\begin{figure}
   	\includegraphics[scale=0.5]{inventario-evento}
\end{figure}
\tiny{Fuente: van Westen en Landslide types and causes}
\end{frame}
%%%%%%%%%%%%%%%%%%%%%%%%%%%%%%%%%%%%%%%%%%%%%%%%%%%%%%%%%%%%%%%%%%%%%%%%%%%%%%%
\begin{frame}
\frametitle{Escala de análisis}
\begin{figure}
   	\includegraphics[scale=0.5]{inventario-escala}
\end{figure}
\tiny{Fuente: Guzzetti (2005)}
\end{frame}
%%%%%%%%%%%%%%%%%%%%%%%%%%%%%%%%%%%%%%%%%%%%%%%%%%%%%%%%%%%%%%%%%%%%%%%%%%%%%%%
\begin{frame}
\frametitle{Cartografía}
\begin{figure}
   	\includegraphics[scale=0.8]{inventario-carto.jpg}
\end{figure}
\tiny{Fuente: Hussin et al., (2016); Clereci et al. (2006)}
\end{frame}
%%%%%%%%%%%%%%%%%%%%%%%%%%%%%%%%%%%%%%%%%%%%%%%%%%%%%%%%%%%%%%%%%%%%%%%%%%%%%%%
\begin{frame}
\frametitle{Actividad}
\begin{figure}
   	\includegraphics[scale=0.5]{menm-actividad}
\end{figure}
\tiny{Fuente: tomado de Cruden \& Varnes (1996)}
\end{frame}
%%%%%%%%%%%%%%%%%%%%%%%%%%%%%%%%%%%%%%%%%%%%%%%%%%%%%%%%%%%%%%%%%%%%%%%%%%%%%%%
\begin{frame}
\frametitle{Activos}
\begin{figure}
   	\includegraphics[scale=0.6]{activos}
\end{figure}
\tiny{Fuente: McCalpin (1984)}
\end{frame}
%%%%%%%%%%%%%%%%%%%%%%%%%%%%%%%%%%%%%%%%%%%%%%%%%%%%%%%%%%%%%%%%%%%%%%%%%%%%%%%
\begin{frame}
\frametitle{Inactivo joven}
\begin{figure}
   	\includegraphics[scale=0.6]{inactivo}
\end{figure}
\tiny{Fuente: McCalpin (1984)}
\end{frame}
%%%%%%%%%%%%%%%%%%%%%%%%%%%%%%%%%%%%%%%%%%%%%%%%%%%%%%%%%%%%%%%%%%%%%%%%%%%%%%%
\begin{frame}
\frametitle{Inactivo maduro}
\begin{figure}
   	\includegraphics[scale=0.6]{maduro}
\end{figure}
\tiny{Fuente: McCalpin (1984)}
\end{frame}
%%%%%%%%%%%%%%%%%%%%%%%%%%%%%%%%%%%%%%%%%%%%%%%%%%%%%%%%%%%%%%%%%%%%%%%%%%%%%%%
\begin{frame}
\frametitle{Inactivo antiguo}
\begin{figure}
   	\includegraphics[scale=0.6]{old}
\end{figure}
\tiny{Fuente: McCalpin (1984)}
\end{frame}
%%%%%%%%%%%%%%%%%%%%%%%%%%%%%%%%%%%%%%%%%%%%%%%%%%%%%%%%%%%%%%%%%%%%%%%%%%%%%%%
\begin{frame}
\frametitle{Errores}
\begin{figure}
   	\includegraphics[scale=0.45]{inventario-error}
\end{figure}
\tiny{Fuente: Santagelo et al. (2015)}
\end{frame}
%%%%%%%%%%%%%%%%%%%%%%%%%%%%%%%%%%%%%%%%%%%%%%%%%%%%%%%%%%%%%%%%%%%%%%%%%%%%%%%
\begin{frame}
\frametitle{Subregistro}
\justifying
\scriptsize{Los inventarios de deslizamientos tienden a ser sesgados por la relación entre magnitud-frecuencia en los deslizamientos asociada a:
\begin{itemize}
\item Se desconoce la longitud del periodo de observación.
\item Se desconoce el papel de eventos detonantes de alta intensidad tales como sismos.
\item La geometría de los pequeños deslizamientos tiende a ser eliminada por procesos erosivos y la vegetación.
\end{itemize}
}
\begin{figure}
   	\includegraphics[scale=0.45]{inventario-subregistro}
\end{figure}
\tiny{Fuente: Fuente: Korup (2005)}
\end{frame}
%%%%%%%%%%%%%%%%%%%%%%%%%%%%%%%%%%%%%%%%%%%%%%%%%%%%%%%%%%%%%%%%%%%%%%%%%%%%%%%
\begin{frame}
\frametitle{Errores}
\begin{figure}
   	\includegraphics[scale=0.6]{inventario-digitalizacion}
\end{figure}
\tiny{Fuente: Santagelo et al. (2015)}
\end{frame}
%%%%%%%%%%%%%%%%%%%%%%%%%%%%%%%%%%%%%%%%%%%%%%%%%%%%%%%%%%%%%%%%%%%%%%%%%%%%%%%
\begin{frame}
\frametitle{Ejemplos}
\framesubtitle{DEM}
\begin{figure}
   	\includegraphics[scale=0.5]{dtm5}
\end{figure}
\end{frame}
%%%%%%%%%%%%%%%%%%%%%%%%%%%%%%%%%%%%%%%%%%%%%%%%%%%%%%%%%%%%%%%%%%%%%%%%%%%%%%%
\begin{frame}
\frametitle{Ejemplos}
\framesubtitle{DEM}
\begin{figure}
   	\includegraphics[scale=0.5]{dtm6}
\end{figure}
\end{frame}
%%%%%%%%%%%%%%%%%%%%%%%%%%%%%%%%%%%%%%%%%%%%%%%%%%%%%%%%%%%%%%%%%%%%%%%%%%%%%%%
\begin{frame}
\frametitle{Ejemplos}
\framesubtitle{LIDAR}
\begin{figure}
   	\includegraphics[scale=0.65]{inventario-lidar}
\end{figure}
\end{frame}
%%%%%%%%%%%%%%%%%%%%%%%%%%%%%%%%%%%%%%%%%%%%%%%%%%%%%%%%%%%%%%%%%%%%%%%%%%%%%%%
\begin{frame}
\frametitle{Ejemplos}
\framesubtitle{LIDAR}
\begin{figure}
   	\includegraphics[scale=0.65]{inventario-lidar2}
\end{figure}
\end{frame}
%%%%%%%%%%%%%%%%%%%%%%%%%%%%%%%%%%%%%%%%%%%%%%%%%%%%%%%%%%%%%%%%%%%%%%%%%%%%%%%
\begin{frame}
\frametitle{Ejemplos}
\framesubtitle{NDVI}
\begin{figure}
   	\includegraphics[scale=0.62]{inventario-ndvi}
\end{figure}
\end{frame}
%%%%%%%%%%%%%%%%%%%%%%%%%%%%%%%%%%%%%%%%%%%%%%%%%%%%%%%%%%%%%%%%%%%%%%%%%%%%%%%
\begin{frame}
\frametitle{Características en fotos}
\begin{figure}
   	\includegraphics[scale=0.45]{inventario-carac}
\end{figure}
\end{frame}
%%%%%%%%%%%%%%%%%%%%%%%%%%%%%%%%%%%%%%%%%%%%%%%%%%%%%%%%%%%%%%%%%%%%%%%%%%%%%%%
\begin{frame}
\frametitle{Características en fotos}
\begin{figure}
   	\includegraphics[scale=0.62]{inventario-caract2}
\end{figure}
\end{frame}
%%%%%%%%%%%%%%%%%%%%%%%%%%%%%%%%%%%%%%%%%%%%%%%%%%%%%%%%%%%%%%%%%%%%%%%%%%%%%%%
\begin{frame}
\frametitle{Características en fotos}
\begin{figure}
   	\includegraphics[scale=0.6]{inventario-caract3}
\end{figure}
\end{frame}
%%%%%%%%%%%%%%%%%%%%%%%%%%%%%%%%%%%%%%%%%%%%%%%%%%%%%%%%%%%%%%%%%%%%%%%%%%%%%%%
\begin{frame}
\begin{figure}
   	\includegraphics[scale=0.37]{logo3}
\end{figure}
\end{frame}
%%%%%%%%%%%%%%%%%%%%%%%%%%%%%%%%%%%%%%%%%%%%%%%%%%%%%%%%%%%%%%%%%%%%%%%%%%%%%%%
\end{document}