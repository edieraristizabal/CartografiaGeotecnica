\documentclass[12pt]{beamer}

\usepackage{beamerthemeCambridgeUS}
\usepackage{textpos}
\usepackage{ragged2e}

\graphicspath{{G:/My Drive/FIGURAS/}}

\title[Introducción]{CARTOGRAFÍA GEOTÉCNICA}
\author[Edier Aristizábal]{Edier V. Aristizábal G.}
\institute{\emph{evaristizabalg@unal.edu.co}}
\date{Version:\today}

\addtobeamertemplate{headline}{}{%
	\begin{textblock*}{2mm}(.9\textwidth,0cm)
	\hfill\includegraphics[height=1cm]{un}  
	\end{textblock*}}

\begin{document}
%%%%%%%%%%%%%%%%%%%%%%%%%%%%%%%%%%%%%%%%%%%%%%%%%%%%%%%%%%%%%%%%
\begin{frame}
\titlepage
\centering
\includegraphics[width=5cm]{unal}\hspace*{4.75cm}~%
\includegraphics[width=2cm]{logo3} 
\end{frame}
%%%%%%%%%%%%%%%%%%%%%%%%%%%%%%%%%%%%%%%%%%%%%%%%%%%%%%%%%%%%%%%%
\begin{frame}
\frametitle{Curso: Cartografía Geotécnica}
\scriptsize{La información del curso, tales como programa, presentaciones, lecturas recomendadas, talleres y demás podrá ser consultado en \emph{Google Classroom}:}
\vfill
\centering
\url{https://classroom.google.com/u/0/c/NjI1NzExOTU0MTla}\\
Class code: [pfi7tsd]
\includegraphics[width=10cm]{carto-geotecnica}
\end{frame}
%%%%%%%%%%%%%%%%%%%%%%%%%%%%%%%%%%%%%%%%%%%%%%%%%%%%%%%%%%%%%%%%%
\begin{frame}
\frametitle{Objetivos y alcance del curso}
\scriptsize{
\justifying
El curso de \textbf{Cartografía Geotécnica} está orientado para estudiantes de ingeniería que deseen  adquirir conocimientos sobre evaluación de la susceptibilidad y amenaza por movimientos en masa en zonas de montaña, con enfoque en ordenamiento territorial. 
\vfill
El curso es teórico - práctico, en el cual se hace una revisión detallada del estado del arte y las diferentes metodologías de zonificación utilizadas alrededor del mundo, incluyendo métodos de aprendizaje automático (\emph{machine learning}), minería de datos (\emph{data mining}), y análisis espacial de datos (\emph{big data}).
\vfill
El alcance de este curso es aprender a cartografiar y construir mapas de susceptibilidad y amenaza como herramienta para el ordenamiento del territorio.
\vfill
Van a adquirir herramientas prácticas que seguramente abrirán sus oportunidades laborales en el campo de la geología aplicada a la ingeniería.
\vfill
Vamos a trabajar con herramientas como QGIS y Python…pero no es un curso sobre estas herramientas. El profesor no es un experto en dichas herramientas,  ni le interesa serlo, es simplemente un usuario.
\vfill
\centering
\textbf{Este curso es muy fácil de ganar…pero hay que trabajar mucho.}

}
\end{frame}
%%%%%%%%%%%%%%%%%%%%%%%%%%%%%%%%%%%%%%%%%%%%%%%%%%%%%%%%%%%%%%%%%
\begin{frame}
\frametitle{Modelo experimental de Kolb}
\centering
\includegraphics[width=7cm]{kolb}
\end{frame}
%%%%%%%%%%%%%%%%%%%%%%%%%%%%%%%%%%%%%%%%%%%%%%%%%%%%%%%%%%%%%%%%%
\begin{frame}
\frametitle{Cronograma y contenido}
\centering
\includegraphics[width=10cm]{cronograma-carto}
\end{frame}
%%%%%%%%%%%%%%%%%%%%%%%%%%%%%%%%%%%%%%%%%%%%%%%%%%%%%%%%%%%%%%%%%
\begin{frame}
\frametitle{Cronograma y contenido}
\centering
\includegraphics[width=10cm]{cronograma-carto1}
\end{frame}
%%%%%%%%%%%%%%%%%%%%%%%%%%%%%%%%%%%%%%%%%%%%%%%%%%%%%%%%%%%%%%%%%
\begin{frame}
\frametitle{Evaluación}
\centering
\includegraphics[width=8cm]{evaluacion-carto}
\end{frame}
%%%%%%%%%%%%%%%%%%%%%%%%%%%%%%%%%%%%%%%%%%%%%%%%%%%%%%%%%%%%%%%%%
\begin{frame}
\centering
\includegraphics[width=11cm]{abstract-carto}
\end{frame}
%%%%%%%%%%%%%%%%%%%%%%%%%%%%%%%%%%%%%%%%%%%%%%%%%%%%%%%%%%%%%%%%%
\end{document}