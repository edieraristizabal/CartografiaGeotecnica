\documentclass[12pt]{beamer}

\usepackage{beamerthemeCambridgeUS}
\usepackage{textpos}
\usepackage{ragged2e}

\graphicspath{{G:/My Drive/FIGURAS/}}

\title[Hidrología \& Estabilidad de laderas]{CARTOGRAFÍA GEOTÉCNICA}
\author[Edier Aristizábal]{Edier V. Aristizábal G.}
\institute{\emph{evaristizabalg@unal.edu.co}}
\date{Version:\today}

\addtobeamertemplate{headline}{}{%
	\begin{textblock*}{2mm}(.9\textwidth,0cm)
	\hfill\includegraphics[height=1cm]{un}  
	\end{textblock*}}

\begin{document}
%%%%%%%%%%%%%%%%%%%%%%%%%%%%%%%%%%%%%%%%%%%%%%%%%%%%%%%%%%%%%%%%
\begin{frame}
\titlepage
\centering
\includegraphics[width=5cm]{unal}\hspace*{4.75cm}~%
\includegraphics[width=2cm]{logo3} 
\end{frame}
%%%%%%%%%%%%%%%%%%%%%%%%%%%%%%%%%%%%%%%%%%%%%%%%%%%%%%%%%%%%%%%%
\begin{frame}
\frametitle{Laderas (\emph{Hillslopes})}
\scriptsize{
\begin{itemize}
\item “Elemento básico de todo paisaje” {\tiny Finlayson \& Statham (1980)}.
\vfill
\item Gran variedad de tamaños y formas.
\vfill
\item Resultado de los procesos de laderas.
\vfill
\item Unidad de geoforma inclinada con un ángulo de pendiente mayor que un umbral mínimo que lo delimita de llanuras y menor a un umbral máximo que lo delimita de superficies verticales, y que es limitado por una unidad de geoforma en la parte superior e inferior {\tiny (Denh et al., 2001)}.
\end{itemize}
}
\includegraphics[scale=0.53]{ladera}
\end{frame}
%%%%%%%%%%%%%%%%%%%%%%%%%%%%%%%%%%%%%%%%%%%%%%%%%%%%%%%%%%%%%%%%%
\begin{frame}
\frametitle{Formación de suelo}
\begin{figure}
\centering
\includegraphics[scale=0.53]{espesor-suelo}
\end{figure}
\tiny{Fuente: Notas de clase Introductory Geomorphology del Prof. D. Montgomery }
\end{frame}
%%%%%%%%%%%%%%%%%%%%%%%%%%%%%%%%%%%%%%%%%%%%%%%%%%%%%%%%%%%%%%%%%
\begin{frame}
\frametitle{Evolución de la ladera}
\begin{figure}
\centering
\includegraphics[scale=0.7]{davispenck1}
\end{figure}
\tiny{Fuente: Notas de clase Introductory Geomorphology del Prof. D. Montgomery }
\end{frame}
%%%%%%%%%%%%%%%%%%%%%%%%%%%%%%%%%%%%%%%%%%%%%%%%%%%%%%%%%%%%%%%%%
\begin{frame}
\frametitle{Modelo de unidad de la ladera}
\begin{figure}
\centering
\includegraphics[scale=0.6]{ladera1}
\end{figure}
\tiny{Fuente: Dalrymple et al (1969)}
\end{frame}
%%%%%%%%%%%%%%%%%%%%%%%%%%%%%%%%%%%%%%%%%%%%%%%%%%%%%%%%%%%%%%%%%
\begin{frame}
\frametitle{Procesos de ladera}
\begin{figure}
\centering
\includegraphics[scale=0.6]{ladera-procesos}
\end{figure}
\end{frame}
%%%%%%%%%%%%%%%%%%%%%%%%%%%%%%%%%%%%%%%%%%%%%%%%%%%%%%%%%%%%%%%%%
\begin{frame}
\frametitle{Tipos de suelo}
\begin{figure}
\centering
\includegraphics[scale=0.5]{suelos-tipo}
\end{figure}
\end{frame}
%%%%%%%%%%%%%%%%%%%%%%%%%%%%%%%%%%%%%%%%%%%%%%%%%%%%%%%%%%%%%%%%%
\begin{frame}
\frametitle{Definición: Perfil de suelo}
\scriptsize{Se entiende por perfil de un suelo a la sección vertical de un terreno, constituido por una secuencia de horizontes o capas, separables por sus características morfológicas, físicas, químicas y mineralógicas. }
\begin{figure}
\centering
\includegraphics[scale=0.5]{perfil-suelo}
\end{figure}
\end{frame}
%%%%%%%%%%%%%%%%%%%%%%%%%%%%%%%%%%%%%%%%%%%%%%%%%%%%%%%%%%%%%%%%%
\begin{frame}
\frametitle{Perfil de suelo}
\begin{figure}
\centering
\includegraphics[scale=0.5]{suelosinsitu1}
\end{figure}
\end{frame}
%%%%%%%%%%%%%%%%%%%%%%%%%%%%%%%%%%%%%%%%%%%%%%%%%%%%%%%%%%%%%%%%%
\begin{frame}
\frametitle{Las laderas falla por cizalla...}
\begin{figure}
\centering
\includegraphics[scale=0.5]{ladera-cizalla}
\end{figure}
\tiny{Notas del prof. Edwin García}
\end{frame}
%%%%%%%%%%%%%%%%%%%%%%%%%%%%%%%%%%%%%%%%%%%%%%%%%%%%%%%%%%%%%%%%%
\begin{frame}
\frametitle{Círculos de Mohr}
\begin{figure}
\centering
\includegraphics[scale=0.5]{mohr}
\end{figure}
\tiny{Notas del prof. Edwin García}
\end{frame}
%%%%%%%%%%%%%%%%%%%%%%%%%%%%%%%%%%%%%%%%%%%%%%%%%%%%%%%%%%%%%%%%%
\begin{frame}
\frametitle{Círculos de Mohr}
\begin{figure}
\centering
\includegraphics[scale=0.5]{mohr1}
\end{figure}
\tiny{Notas del prof. Edwin García}
\end{frame}
%%%%%%%%%%%%%%%%%%%%%%%%%%%%%%%%%%%%%%%%%%%%%%%%%%%%%%%%%%%%%%%%%
\begin{frame}
\frametitle{Círculos de Mohr}
\scriptsize{Coulomb (1776) $\rightarrow$ resistencia al corte como función de la cohesión y la fricción}
\begin{figure}
\centering
\includegraphics[scale=0.5]{mohr5}
\end{figure}
\tiny{Notas del prof. Edwin García}
\end{frame}
%%%%%%%%%%%%%%%%%%%%%%%%%%%%%%%%%%%%%%%%%%%%%%%%%%%%%%%%%%%%%%%%%
\begin{frame}
\frametitle{Círculos de Mohr}
\begin{figure}
\centering
\includegraphics[scale=0.5]{mohr2}
\end{figure}
\tiny{Notas del prof. Edwin García}
\end{frame}
%%%%%%%%%%%%%%%%%%%%%%%%%%%%%%%%%%%%%%%%%%%%%%%%%%%%%%%%%%%%%%%%%
\begin{frame}
\frametitle{Círculos de Mohr}
\begin{figure}
\centering
\includegraphics[scale=0.5]{mohr3}
\end{figure}
\tiny{Notas del prof. Edwin García}
\end{frame}
%%%%%%%%%%%%%%%%%%%%%%%%%%%%%%%%%%%%%%%%%%%%%%%%%%%%%%%%%%%%%%%%%
\begin{frame}
\frametitle{Círculos de Mohr}
\scriptsize{Mohr (1900) afirmó que un material falla debido a una combinación crítica de esfuerzo normal y cortante, y no solo por la presencia de uno de ellos en su estado máximo}
\begin{figure}
\centering
\includegraphics[scale=0.48]{mohr4}
\end{figure}
\tiny{Notas del prof. Edwin García}
\end{frame}
%%%%%%%%%%%%%%%%%%%%%%%%%%%%%%%%%%%%%%%%%%%%%%%%%%%%%%%%%%%%%%%%%
\begin{frame}
\frametitle{Ciclo hidrológico}
\begin{figure}
\centering
\includegraphics[scale=0.44]{ciclo}
\end{figure}
\end{frame}
%%%%%%%%%%%%%%%%%%%%%%%%%%%%%%%%%%%%%%%%%%%%%%%%%%%%%%%%%%%%%%%%%
\begin{frame}
\frametitle{El agua en el suelo}
\begin{figure}
\centering
\includegraphics[scale=0.48]{aguasuelo}
\end{figure}
\end{frame}
%%%%%%%%%%%%%%%%%%%%%%%%%%%%%%%%%%%%%%%%%%%%%%%%%%%%%%%%%%%%%%%%%
\begin{frame}
\frametitle{El agua en el suelo}
\begin{figure}
\centering
\includegraphics[scale=0.48]{aguasuelo1}
\end{figure}
\end{frame}
%%%%%%%%%%%%%%%%%%%%%%%%%%%%%%%%%%%%%%%%%%%%%%%%%%%%%%%%%%%%%%%%%
\begin{frame}
\frametitle{El agua en el suelo}
\begin{figure}
\centering
\includegraphics[scale=0.48]{vadosa1}
\end{figure}
\end{frame}
%%%%%%%%%%%%%%%%%%%%%%%%%%%%%%%%%%%%%%%%%%%%%%%%%%%%%%%%%%%%%%%%%
\begin{frame}
\frametitle{El agua en el suelo}
\begin{figure}
\centering
\includegraphics[scale=0.48]{vadosa2}
\end{figure}
\end{frame}
%%%%%%%%%%%%%%%%%%%%%%%%%%%%%%%%%%%%%%%%%%%%%%%%%%%%%%%%%%%%%%%%%
\begin{frame}
\frametitle{Esfuerzos efectivos}
\scriptsize{Terzagui (1936) $\rightarrow$ Principio de esfuerzos efectivos}
\begin{figure}
\centering
\includegraphics[scale=0.48]{esfuerzos-efectivos}
\end{figure}
\tiny{Notas del prof. Edwin García}
\end{frame}
%%%%%%%%%%%%%%%%%%%%%%%%%%%%%%%%%%%%%%%%%%%%%%%%%%%%%%%%%%%%%%%%%
\begin{frame}
\frametitle{Esfuerzos efectivos}
\scriptsize{Terzagui (1936) $\rightarrow$ Principio de esfuerzos efectivos}
\begin{figure}
\centering
\includegraphics[scale=0.48]{mohr5}
\end{figure}
\tiny{Notas del prof. Edwin García}
\end{frame}
%%%%%%%%%%%%%%%%%%%%%%%%%%%%%%%%%%%%%%%%%%%%%%%%%%%%%%%%%%%%%%%%%
\begin{frame}
\frametitle{Esfuerzos efectivos \& Totales}
\begin{figure}
\centering
\includegraphics[scale=0.48]{mohr7}
\end{figure}
\tiny{Notas del prof. Edwin García}
\end{frame}
%%%%%%%%%%%%%%%%%%%%%%%%%%%%%%%%%%%%%%%%%%%%%%%%%%%%%%%%%%%%%%%%%
\begin{frame}
\frametitle{Esfuerzos efectivos \& Totales}
\begin{figure}
\centering
\includegraphics[scale=0.48]{mohr8}
\end{figure}
\tiny{Notas del prof. Edwin García}
\end{frame}
%%%%%%%%%%%%%%%%%%%%%%%%%%%%%%%%%%%%%%%%%%%%%%%%%%%%%%%%%%%%%%%%%
\begin{frame}
\frametitle{Permeabilidad}
\begin{figure}
\centering
\includegraphics[scale=0.45]{permeabilidad}
\end{figure}
\end{frame}
%%%%%%%%%%%%%%%%%%%%%%%%%%%%%%%%%%%%%%%%%%%%%%%%%%%%%%%%%%%%%%%%%
\begin{frame}
\frametitle{Permeabilidad}
\begin{figure}
\centering
\includegraphics[scale=0.45]{macroporos}
\end{figure}
\end{frame}
%%%%%%%%%%%%%%%%%%%%%%%%%%%%%%%%%%%%%%%%%%%%%%%%%%%%%%%%%%%%%%%%%
\begin{frame}
\frametitle{Flujo preferencial}
\begin{figure}
\centering
\includegraphics[scale=0.5]{flujo-preferencial}
\end{figure}
\tiny{Fuente: Sidle \& Bogaard (2016)}
\end{frame}
%%%%%%%%%%%%%%%%%%%%%%%%%%%%%%%%%%%%%%%%%%%%%%%%%%%%%%%%%%%%%%%%%
\begin{frame}
\frametitle{Análisis de talud infinito}
\begin{figure}
\centering
\includegraphics[scale=0.45]{taludinfinito}
\end{figure}
\end{frame}
%%%%%%%%%%%%%%%%%%%%%%%%%%%%%%%%%%%%%%%%%%%%%%%%%%%%%%%%%%%%%%%%%
\begin{frame}
\frametitle{Mecanismo de falla por lluvia}
\scriptsize{
\begin{itemize}
\item Incremento de presiones de poros positivas que genera el proceso de licuefacción del material generando movimientos tipo flujos.
\item Reducción de las presiones de poros negativas en condiciones no saturadas donde la falla ocurre por reducción de la succión y la masa en desplazamiento se comporta como un cuerpo semi-rigido, deslizamientos.
\end{itemize}
}
\begin{figure}
\centering
\includegraphics[scale=0.55]{ladera-lluvia}
\end{figure}
\tiny{Fuente: Collins \& Znidarcic (2004)}
\end{frame}
%%%%%%%%%%%%%%%%%%%%%%%%%%%%%%%%%%%%%%%%%%%%%%%%%%%%%%%%%%%%%%%%%
\begin{frame}
\frametitle{Efectos de la vegetación}
\begin{figure}
\centering
\includegraphics[scale=0.48]{vegetacion-suelos}
\end{figure}
\end{frame}
%%%%%%%%%%%%%%%%%%%%%%%%%%%%%%%%%%%%%%%%%%%%%%%%%%%%%%%%%%%%%%%%%
\begin{frame}
\frametitle{Efectos de la vegetación}
\begin{figure}
\centering
\includegraphics[scale=0.5]{vegetacion-suelos1}
\end{figure}
\end{frame}
%%%%%%%%%%%%%%%%%%%%%%%%%%%%%%%%%%%%%%%%%%%%%%%%%%%%%%%%%%%%%%%%%

\end{document}