\documentclass[12pt]{beamer}

\usepackage{beamerthemeCambridgeUS}
\usepackage{textpos}
\usepackage{ragged2e}

\graphicspath{{G:/My Drive/FIGURAS/}}

\title[Factor detonante]{CARTOGRAFÍA GEOTÉCNICA}
\author[Edier Aristizábal]{Edier V. Aristizábal G.}
\institute{\emph{evaristizabalg@unal.edu.co}}
\date{Version:\today}

\addtobeamertemplate{headline}{}{%
	\begin{textblock*}{2mm}(.9\textwidth,0cm)
	\hfill\includegraphics[height=1cm]{un}  
	\end{textblock*}}

\begin{document}
%%%%%%%%%%%%%%%%%%%%%%%%%%%%%%%%%%%%%%%%%%%%%%%%%%%%%%%%%%%%%%%%
\begin{frame}
\titlepage
\centering
\includegraphics[width=5cm]{unal}\hspace*{4.75cm}~%
\includegraphics[width=2cm]{logo3} 
\end{frame}
%%%%%%%%%%%%%%%%%%%%%%%%%%%%%%%%%%%%%%%%%%%%%%%%%%%%%%%%%%%%%%%%
\begin{frame}
\frametitle{Antrópico}
\begin{figure}
\centering
\includegraphics[scale=0.47]{menm-antropico} 
\end{figure}
\end{frame}
%%%%%%%%%%%%%%%%%%%%%%%%%%%%%%%%%%%%%%%%%%%%%%%%%%%%%%%%%%%%%%%%
\begin{frame}
\frametitle{Volcanes}
\begin{figure}
\centering
\includegraphics[scale=0.47]{menm-volcanes} 
\caption{Monte Santa Helena (USA), mayo 18 de 1980}
\end{figure}
\end{frame}
%%%%%%%%%%%%%%%%%%%%%%%%%%%%%%%%%%%%%%%%%%%%%%%%%%%%%%%%%%%%%%%%
\begin{frame}
\frametitle{Sismos}
\begin{figure}
\centering
\includegraphics[scale=0.47]{menm-sismos} 
\caption{Sismo de Kumamoto (Japón) abril 16 de 2016, Mw: 7.0}
\end{figure}
\end{frame}
%%%%%%%%%%%%%%%%%%%%%%%%%%%%%%%%%%%%%%%%%%%%%%%%%%%%%%%%%%%%%%%%
\begin{frame}
\frametitle{Precipitación}
\begin{figure}
\centering
\includegraphics[scale=0.47]{menm-lluvia1} 
\caption{Evento de lluvia en la ciudad de Medellín en horas de la tarde del mes de febrero de 2017}
\end{figure}
\end{frame}
%%%%%%%%%%%%%%%%%%%%%%%%%%%%%%%%%%%%%%%%%%%%%%%%%%%%%%%%%%%%%%%%
\begin{frame}
\frametitle{Susceptibilidad vs. Amenaza vs. Riesgo}
\scriptsize{
\begin{itemize}
\item \textbf{Mapas de incidencia espacial}\\
\textbf{Susceptibilidad}: tendencia de un movimiento en masa a ser generado en el futuro en un área específica (Brad, 1984). Posibilidad de que un fenómeno ocurra en un área de acuerdo con las condiciones locales del terreno, y especifican que factores detonantes tales como precipitación o sismicidad no son considerados (Soeters y van Westen, 1996).  Dónde: \emph{Probabilidad espacial}.
\item \textbf{Mapas de incidencia espacio-temporal y pronóstico}\\
\textbf{Amenaza}: probabilidad de ocurrencia de un potencial fenómeno destructivo dentro de un específico período de tiempo y en una determinada área (Varnes, 1984). Dónde (intensidad)? Cuándo (frecuencia), Magnitud (Volumen). \emph{Probabilidad espacial y temporal}.
\item \textbf{Mapas de evaluación de las consecuencias}\\
\textbf{Riesgo}: Evaluación de las potenciales consecuencias en términos de pérdidas humanas y pérdidas económicas. \textbf{Dónde}? (intensidad) \textbf{Cuándo}? (frecuencia) \textbf{Magnitud} (Volumen) \textbf{Cuánto}? (consecuencia)
\end{itemize}
}
\begin{figure}
\centering
\includegraphics[scale=0.6]{amenaza2} 
\end{figure}
\end{frame}
%%%%%%%%%%%%%%%%%%%%%%%%%%%%%%%%%%%%%%%%%%%%%%%%%%%%%%%%%%%%%%%%%%%%%
\begin{frame}
\frametitle{MenM detonados por lluvia}
\begin{figure}
\centering
\includegraphics[scale=0.47]{aguasuelo} 
\end{figure}
\end{frame}
%%%%%%%%%%%%%%%%%%%%%%%%%%%%%%%%%%%%%%%%%%%%%%%%%%%%%%%%%%%%%%%%
\begin{frame}
\frametitle{Cómo se mide la lluvia?}
\framesubtitle{Pluviometros}
\begin{figure}
\centering
\includegraphics[scale=0.7]{raingauge} 
\end{figure}
\end{frame}
%%%%%%%%%%%%%%%%%%%%%%%%%%%%%%%%%%%%%%%%%%%%%%%%%%%%%%%%%%%%%%%%
\begin{frame}
\frametitle{Cómo se estima la lluvia?}
\framesubtitle{Radares}
\begin{figure}
\centering
\includegraphics[scale=0.7]{radar-hidro} 
\includegraphics[scale=0.5]{radar-hidro1} 
\end{figure}
\end{frame}
%%%%%%%%%%%%%%%%%%%%%%%%%%%%%%%%%%%%%%%%%%%%%%%%%%%%%%%%%%%%%%%%
\begin{frame}
\frametitle{Cómo se estima la lluvia?}
\framesubtitle{Satélites}
\begin{figure}
\centering
\includegraphics[scale=0.55]{trmm} 
\end{figure}
\end{frame}
%%%%%%%%%%%%%%%%%%%%%%%%%%%%%%%%%%%%%%%%%%%%%%%%%%%%%%%%%%%%%%%%
\begin{frame}
\frametitle{MenM detonados por lluvia}
\framesubtitle{Definición}
\scriptsize{soil – slip o debris avalanche: movimientos en masa con superficie de falla planar y paralela a la ladera, pequeño espesor  (0,3 – 2 m), área de escarpe pequeña.
}
\begin{figure}
\centering
\includegraphics[scale=0.40]{talud-infinito} 
\end{figure}
\tiny{Fuente: Hungr et al., (2001); Anderson \& Sitar, (1995), Dietrich \& Montgomery (1998); Crosta (1998), Aristizábal et  al.  (2015) }
\end{frame}
%%%%%%%%%%%%%%%%%%%%%%%%%%%%%%%%%%%%%%%%%%%%%%%%%%%%%%%%%%%%%%%%
\begin{frame}
\frametitle{Mecanismo}
\framesubtitle{Avance frente húmedo}
\begin{figure}
\centering
\includegraphics[scale=0.45]{frente-humedo} 
\end{figure}
\end{frame}
%%%%%%%%%%%%%%%%%%%%%%%%%%%%%%%%%%%%%%%%%%%%%%%%%%%%%%%%%%%%%%%%
\begin{frame}
\frametitle{Mecanismo}
\framesubtitle{Formación nivel frático colgado}
\begin{figure}
\centering
\includegraphics[scale=0.45]{nivel-colgado} 
\end{figure}
\end{frame}
%%%%%%%%%%%%%%%%%%%%%%%%%%%%%%%%%%%%%%%%%%%%%%%%%%%%%%%%%%%%%%%%
\begin{frame}
\frametitle{Mecanismo complejo}
\begin{figure}
\centering
\includegraphics[scale=0.6]{saturado} 
\end{figure}
\end{frame}
%%%%%%%%%%%%%%%%%%%%%%%%%%%%%%%%%%%%%%%%%%%%%%%%%%%%%%%%%%%%%%%%
\begin{frame}
\frametitle{Variables}
\framesubtitle{Conductividad hidráulica}
\scriptsize{
\begin{itemize} 
\item Pequeñas variaciones en la conductividad hidráulica controlan la localización de la ocurrencia de movimientos en masa.
\vfill
\item Aunque el suelo sea texturalmente homogéneo, existen variaciones en la conductividad hidráulica debido al contenido de agua y la succión.
\vfill
\item Suelos con bajas permeabilidades son mas estables ante eventos cortos de alta intensidad. 
\vfill
\item Suelos con altas permeabilidades fallan por acumulación del incremento de las presiones de poros.
\vfill
\item Suelos con altas permeabilidades generan movimientos de masa con mayor capacidad destructiva, movimientos mas rápidos y que viajan mayores distancias.
\end{itemize}
}
\vfill
\tiny{Fuente: Rahardjo et al., 2007; Rahimi et al., 2010; Wang \& Sassa, 2003; Li et al., 2005; Setyo \& Liao, 2008; Mukhlisin et al., 2006
}
\end{frame}
%%%%%%%%%%%%%%%%%%%%%%%%%%%%%%%%%%%%%%%%%%%%%%%%%%%%%%%%%%%%%%%%
\begin{frame}
\frametitle{Variables}
\framesubtitle{Morfología de la ladera}
\scriptsize{
\begin{itemize} 
\item El perfil longitudinal (cóncavo – recto – convexo) controla la velocidad y cambio del flujo de agua en el suelo.
\vfill
\item El perfil perpendicular (convergente – paralelo – divergente) controla la convergencia topográfica y juega un papel mas importante. 
\vfill
\item Convexas - Divergentes. Mas estables
\vfill
\item Cóncavas – Convergentes. Menos estables
\end{itemize}
}
\vfill
\begin{figure}
\centering
\includegraphics[scale=0.4]{ladera-concava} 
\end{figure}
\tiny{Fuente: Talebi et al., 2008, Borga et al., 2002; Iida, 1999}
\end{frame}
%%%%%%%%%%%%%%%%%%%%%%%%%%%%%%%%%%%%%%%%%%%%%%%%%%%%%%%%%%%%%%%%
\begin{frame}
\frametitle{Variables}
\framesubtitle{Lluvia antecedente}
\scriptsize{
\begin{itemize} 
\item Existe consenso en cuanto a que para suelos con baja permeabilidad, la lluvia antecedente juega un papel importante .
\vfill
\item Los eventos detonados por condiciones de lluvia antecedente tiene superficies de fallas profundas (> 6m).
\end{itemize}
}
\vfill
\begin{figure}
\centering
\includegraphics[scale=0.4]{ladera-lluvia1} 
\end{figure}
\tiny{Fuente: Talebi et al., 2008, Borga et al., 2002; Iida, 1999}
\end{frame}
%%%%%%%%%%%%%%%%%%%%%%%%%%%%%%%%%%%%%%%%%%%%%%%%%%%%%%%%%%%%%%%%
\begin{frame}
\frametitle{Variables}
\framesubtitle{Intensidad vs Duración}
\scriptsize{
\begin{itemize} 
\item Eventos muy intensos generan movimientos en masa superficiales. 
\vfill
\item Eventos de larga duración e intensidad baja que generan movimientos mucho mas profundos generalmente sobre superficies de fallas pre-existentes. 
\end{itemize}
}
\vfill
\begin{figure}
\centering
\includegraphics[scale=0.4]{ladera-lluvia1} 
\end{figure}
\tiny{Fuente: Gostelow (1991) and Iiritano et al. (1998)}
\end{frame}
%%%%%%%%%%%%%%%%%%%%%%%%%%%%%%%%%%%%%%%%%%%%%%%%%%%%%%%%%%%%%%%%
\begin{frame}
\frametitle{Cuándo ocurren los MenM?}
\framesubtitle{Ciclo diario anual en Colombia}
\begin{figure}
\centering
\includegraphics[scale=0.75]{ciclo-menm} 
\end{figure}
\end{frame}
%%%%%%%%%%%%%%%%%%%%%%%%%%%%%%%%%%%%%%%%%%%%%%%%%%%%%%%%%%%%%%%%
\begin{frame}
\frametitle{Cuándo ocurren los MenM?}
\framesubtitle{Ciclo anual por regiones}
\begin{figure}
\centering
\includegraphics[scale=0.45]{menm-lluvia} 
\end{figure}
\end{frame}
%%%%%%%%%%%%%%%%%%%%%%%%%%%%%%%%%%%%%%%%%%%%%%%%%%%%%%%%%%%%%%%%
\begin{frame}
\frametitle{Cuándo ocurren los MenM?}
\framesubtitle{Ciclo diario anual en el VdeA}
\begin{figure}
\includegraphics[scale=0.5]{menm-lluvia-vdea1} 
\end{figure}
\end{frame}
%%%%%%%%%%%%%%%%%%%%%%%%%%%%%%%%%%%%%%%%%%%%%%%%%%%%%%%%%%%%%%%%
\begin{frame}
\frametitle{Cómo llueve en el Valle de Aburrá?}
\begin{figure}
\centering
\includegraphics[scale=0.45]{lluvia-vdea} 
\end{figure}
\end{frame}
%%%%%%%%%%%%%%%%%%%%%%%%%%%%%%%%%%%%%%%%%%%%%%%%%%%%%%%%%%%%%%%%
\begin{frame}
\frametitle{Cómo llueve en el Valle de Aburrá?}
\begin{figure}
\centering
\includegraphics[scale=0.45]{ciclo-diurno-anual} 
\end{figure}
\end{frame}
%%%%%%%%%%%%%%%%%%%%%%%%%%%%%%%%%%%%%%%%%%%%%%%%%%%%%%%%%%%%%%%%
\begin{frame}
\frametitle{Cómo llueve en el Valle de Aburrá?}
\begin{figure}
\centering
\includegraphics[scale=0.45]{correlacion-espacial-lluvia} 
\end{figure}
\end{frame}
%%%%%%%%%%%%%%%%%%%%%%%%%%%%%%%%%%%%%%%%%%%%%%%%%%%%%%%%%%%%%%%%
\begin{frame}
\frametitle{Factor detonante: Sismo}
\begin{figure}
\centering
\includegraphics[scale=0.45]{menm-sismos1} 
\end{figure}
\end{frame}
%%%%%%%%%%%%%%%%%%%%%%%%%%%%%%%%%%%%%%%%%%%%%%%%%%%%%%%%%%%%%%%%
\begin{frame}
\frametitle{MenM cosísmicos}
\begin{figure}
\centering
\includegraphics[scale=0.45]{menm-sismos2} 
\end{figure}
\end{frame}
%%%%%%%%%%%%%%%%%%%%%%%%%%%%%%%%%%%%%%%%%%%%%%%%%%%%%%%%%%%%%%%%
\begin{frame}
\frametitle{MenM cosísmicos}
\framesubtitle{Distancia al epicentro}
\begin{figure}
\centering
\includegraphics[scale=0.45]{sismos-distancias} 
\end{figure}
\end{frame}
%%%%%%%%%%%%%%%%%%%%%%%%%%%%%%%%%%%%%%%%%%%%%%%%%%%%%%%%%%%%%%%%
\begin{frame}
\frametitle{MenM cosísmicos}
\framesubtitle{Distancia al epicentro}
\begin{figure}
\centering
\includegraphics[scale=0.55]{sismos-distancias1} 
\caption{Sismo de Ludian (China, Agosto 3 de 2014, Mw 6,5)}
\end{figure}
\tiny{Fuente: Zhou et al. (2015)}
\end{frame}
%%%%%%%%%%%%%%%%%%%%%%%%%%%%%%%%%%%%%%%%%%%%%%%%%%%%%%%%%%%%%%%%
\begin{frame}
\frametitle{MenM cosísmicos}
\framesubtitle{Tipo de falla}
\scriptsize{Los movimientos en masa se distribuyeron en una zona mas amplia a lo largo del la falla inversa que en la parte de rumbo.}
\begin{figure}
\centering
\includegraphics[scale=0.5]{sismo-falla} 
\caption{Sismo de Sichuan (China), Mayo 12 de 2008, Ms 8.0)}
\end{figure}
\tiny{Fuente: LARAM 2015 (Prof. Run-qiu Huang) }
\end{frame}
%%%%%%%%%%%%%%%%%%%%%%%%%%%%%%%%%%%%%%%%%%%%%%%%%%%%%%%%%%%%%%%%
\begin{frame}
\frametitle{MenM cosísmicos}
\framesubtitle{Tipo de falla}
\begin{figure}
\centering
\includegraphics[scale=0.55]{sismo-falla1} 
\end{figure}
\tiny{Fuente: LARAM 2015 (Prof. Run-qiu Huang) }
\end{frame}
%%%%%%%%%%%%%%%%%%%%%%%%%%%%%%%%%%%%%%%%%%%%%%%%%%%%%%%%%%%%%%%%
\begin{frame}
\frametitle{Efecto \emph{Hanging-wall}}
\begin{figure}
\centering
\includegraphics[scale=0.50]{hanging-wall} 
\end{figure}
\tiny{Fuente: LARAM 2015 (Prof. Run-qiu Huang) }
\end{frame}
%%%%%%%%%%%%%%%%%%%%%%%%%%%%%%%%%%%%%%%%%%%%%%%%%%%%%%%%%%%%%%%%
\begin{frame}
\frametitle{Efecto dirección}
\begin{figure}
\centering
\includegraphics[scale=0.50]{sismo-direccion} 
\end{figure}
\tiny{Fuente: LARAM 2015 (Prof. Run-qiu Huang) }
\end{frame}
%%%%%%%%%%%%%%%%%%%%%%%%%%%%%%%%%%%%%%%%%%%%%%%%%%%%%%%%%%%%%%%%
\begin{frame}
\frametitle{Efecto dirección}
\begin{figure}
\centering
\includegraphics[scale=0.50]{sismo-direccion1} 
\end{figure}
\tiny{Fuente: LARAM 2015 (Prof. Run-qiu Huang) }
\end{frame}
%%%%%%%%%%%%%%%%%%%%%%%%%%%%%%%%%%%%%%%%%%%%%%%%%%%%%%%%%%%%%%%%
\begin{frame}
\frametitle{Efecto dirección}
\begin{figure}
\centering
\includegraphics[scale=0.55]{sismo-direccion2} 
\end{figure}
\tiny{Fuente: LARAM 2015 (Prof. Run-qiu Huang) }
\end{frame}
%%%%%%%%%%%%%%%%%%%%%%%%%%%%%%%%%%%%%%%%%%%%%%%%%%%%%%%%%%%%%%%%
\begin{frame}
\frametitle{Efecto pendiente}
\begin{figure}
\centering
\includegraphics[scale=0.45]{sismo-pendiente} 
\end{figure}
\tiny{Fuente: LARAM 2015 (Prof. Run-qiu Huang) }
\end{frame}
%%%%%%%%%%%%%%%%%%%%%%%%%%%%%%%%%%%%%%%%%%%%%%%%%%%%%%%%%%%%%%%%
\begin{frame}
\frametitle{Efecto forma de la pendiente}
\begin{figure}
\centering
\includegraphics[scale=0.55]{sismo-forma} 
\end{figure}
\tiny{Fuente: LARAM 2015 (Prof. Run-qiu Huang) }
\end{frame}
%%%%%%%%%%%%%%%%%%%%%%%%%%%%%%%%%%%%%%%%%%%%%%%%%%%%%%%%%%%%%%%%
\begin{frame}
\frametitle{Efecto forma de la pendiente}
\begin{figure}
\centering
\includegraphics[scale=0.55]{sismo-forma1} 
\end{figure}
\tiny{Fuente: LARAM 2015 (Prof. Run-qiu Huang) }
\end{frame}
%%%%%%%%%%%%%%%%%%%%%%%%%%%%%%%%%%%%%%%%%%%%%%%%%%%%%%%%%%%%%%%%
\begin{frame}
\frametitle{Efecto geología y tipo de MenM}
\begin{figure}
\centering
\includegraphics[scale=0.35]{sismo-geologia-menm} 
\end{figure}
\tiny{Fuente: Higaki \& Abe (2013)}
\end{frame}
%%%%%%%%%%%%%%%%%%%%%%%%%%%%%%%%%%%%%%%%%%%%%%%%%%%%%%%%%%%%%%%%
\begin{frame}
\frametitle{Efecto geología y tipo de MenM}
\begin{figure}
\centering
\includegraphics[scale=0.35]{sismo-geologia-menm1} 
\end{figure}
\tiny{Fuente: Higaki \& Abe (2013)}
\end{frame}
%%%%%%%%%%%%%%%%%%%%%%%%%%%%%%%%%%%%%%%%%%%%%%%%%%%%%%%%%%%%%%%%
\begin{frame}
\frametitle{Efecto tipo de MenM}
\begin{figure}
\centering
\includegraphics[scale=0.6]{sismo-tipo-menm} 
\end{figure}
\tiny{Fuente: Tiwari \& Beena (2017)}
\end{frame}
%%%%%%%%%%%%%%%%%%%%%%%%%%%%%%%%%%%%%%%%%%%%%%%%%%%%%%%%%%%%%%%%
\end{document}