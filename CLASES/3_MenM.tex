\documentclass[12pt]{beamer}

\usepackage{beamerthemeCambridgeUS}
\usepackage{textpos}
\usepackage{ragged2e}

\graphicspath{{G:/My Drive/FIGURAS/}}

\title[Movimientos en masa]{CARTOGRAFÍA GEOTÉCNICA}
\author[Edier Aristizábal]{Edier V. Aristizábal G.}
\institute{\emph{evaristizabalg@unal.edu.co}}
\date{Version:\today}

\addtobeamertemplate{headline}{}{%
	\begin{textblock*}{2mm}(.9\textwidth,0cm)
	\hfill\includegraphics[height=1cm]{un}  
	\end{textblock*}}

\begin{document}
%%%%%%%%%%%%%%%%%%%%%%%%%%%%%%%%%%%%%%%%%%%%%%%%%%%%%%%%%%%%%%%%
\begin{frame}
\titlepage
\centering
\includegraphics[width=5cm]{unal}\hspace*{4.75cm}~%
\includegraphics[width=2cm]{logo3} 
\end{frame}
%%%%%%%%%%%%%%%%%%%%%%%%%%%%%%%%%%%%%%%%%%%%%%%%%%%%%%%%%%%%%%%%
\begin{frame}
\begin{figure}
\centering
\includegraphics[scale=0.5]{menm-variedad} 
\end{figure}
\tiny{Tomado de Guzzetti (2005)}
\end{frame}
%%%%%%%%%%%%%%%%%%%%%%%%%%%%%%%%%%%%%%%%%%%%%%%%%%%%%%%%%%%%%%%%
\begin{frame}
\frametitle{Términos utilizados}
\begin{figure}
\centering
\includegraphics[scale=0.5]{terminosMenM} 
\end{figure}
\end{frame}
%%%%%%%%%%%%%%%%%%%%%%%%%%%%%%%%%%%%%%%%%%%%%%%%%%%%%%%%%%%%%%%%
\begin{frame}
\frametitle{Definiciones}
\scriptsize{
\textbf{Varnes (1978)} también de la Escuela Norteamericana considera el término slope movement más indicado y lo define como ¨Movimiento hacia abajo y afuera de los materiales que conforman una ladera bajo la influencia de la gravedad¨
\vfill
\textbf{Brunsden (1984)} de la Escuela Europea utiliza el término mass movement  de acuerdo a Penk (1894), quien lo diferencia del término mass transport por ser un proceso bajo la influencia de la gravedad que no implica un medio de transporte como agua, aire o nieve.
\vfill
\textbf{Sheidegger (1998)} los define como modificaciones del terreno dentro del ciclo geomorfológico continuo, y que corresponden a la respuesta normal del sistema debido a complejos parámetros exogénicos (meteóricos) y endogénicos (tectónicos).
\vfill
\textbf{Cruden (1991)} de la Escuela Norteamericana presenta la definición mas simple acogida oficialmente por la WP/WLI de la UNESCO, utiliza el término landslide y lo define como: ¨Movimiento de una masa de roca, escombros o tierra ladera abajo”. Sin embargo en teoría sólo aplicaría para movimientos que se deslizan sobre una superficie.
}
\end{frame}
%%%%%%%%%%%%%%%%%%%%%%%%%%%%%%%%%%%%%%%%%%%%%%%%%%%%%%%%%%%%%%%%%
\begin{frame}
\frametitle{Definiciones}
\scriptsize{
\justifying
\textbf{Soeters y van Westen (1996)} definen los procesos de movimiento en masa como el resultado de las condiciones naturales del terreno, tales como geomorfología, hidrología y geología, y las modificaciones de estas condiciones por procesos geodinámicos, vegetación, usos del suelo y actividades humanas. Dichas modificaciones activan movimientos lentos, generalmente imperceptibles debido a que las propiedades mecánicas del material o condiciones de equilibrio decrecen gradualmente, y posteriormente, factores como precipitación, sismicidad o cortes de origen antrópico detonan dichos movimientos lentos en rápidos movimientos en masa .
\vfill
\begin{figure}
\centering
\includegraphics[scale=0.7]{MenM} 
\end{figure}
}
\end{frame}
%%%%%%%%%%%%%%%%%%%%%%%%%%%%%%%%%%%%%%%%%%%%%%%%%%%%%%%%%%%%%%%%%
\begin{frame}
\frametitle{Variables condicionantes y detonantes}
\scriptsize{
\justifying
\textbf{Variables condicionantes (preparatorias, cuasi-estaticas)}: las cuales hacen la ladera susceptible a fallar sin siquiera iniciarse  y sin embargo tienden a ubicar la ladera en un estado estable marginal: geología, pendiente, aspecto, elevación, propiedades geotécnicas del suelo, vegetación, y patrones de drenaje de largo plazo y meteorización.
\vfill
\textbf{Variables detonantes (dinámicas)}: las cuales cambian la ladera de una estabilidad marginal a un estado inestable y por lo tanto iniciando una falla en un área de determinada susceptibilidad, tales como lluvias intensas, sismos, deshielo, intervención antrópica. 
\vfill
\begin{figure}
\centering
\includegraphics[scale=0.6]{menm1} 
\end{figure}
}
\tiny{Tomado de Crozier (1989)}
\end{frame}
%%%%%%%%%%%%%%%%%%%%%%%%%%%%%%%%%%%%%%%%%%%%%%%%%%%%%%%%%%%%%%%%%
\begin{frame}
\frametitle{Variables}
\begin{figure}
\centering
\includegraphics[scale=0.5]{menm-variables} 
\end{figure}
\tiny{Modificado de Brunsden (2002)}
\end{frame}
%%%%%%%%%%%%%%%%%%%%%%%%%%%%%%%%%%%%%%%%%%%%%%%%%%%%%%%%%%%%%%%%
\begin{frame}
\frametitle{Partes}
\begin{figure}
\centering
\includegraphics[scale=0.8]{menm-partes} 
\end{figure}
\end{frame}
%%%%%%%%%%%%%%%%%%%%%%%%%%%%%%%%%%%%%%%%%%%%%%%%%%%%%%%%%%%%%%%%
\begin{frame}
\frametitle{Partes}
\begin{figure}
\centering
\includegraphics[scale=0.8]{flujo-partes} 
\end{figure}
\end{frame}
%%%%%%%%%%%%%%%%%%%%%%%%%%%%%%%%%%%%%%%%%%%%%%%%%%%%%%%%%%%%%%%%
\begin{frame}
\frametitle{Clasificaciones}
\begin{figure}
\centering
\includegraphics[scale=0.8]{menm-clasificaciones} 
\end{figure}
\end{frame}
%%%%%%%%%%%%%%%%%%%%%%%%%%%%%%%%%%%%%%%%%%%%%%%%%%%%%%%%%%%%%%%%
\begin{frame}
\frametitle{Cruden \& Varnes (1996)}
\begin{figure}
\centering
\includegraphics[scale=0.6]{menm-clasificacion} 
\end{figure}
\tiny{Fuente: Cruden \& Varnes (1996)}
\end{frame}
%%%%%%%%%%%%%%%%%%%%%%%%%%%%%%%%%%%%%%%%%%%%%%%%%%%%%%%%%%%%%%%%
\begin{frame}
\frametitle{Cruden \& Varnes (1996)}
\begin{figure}
\centering
\includegraphics[scale=0.5]{cruden-varnes} 
\end{figure}
\tiny{Fuente: Cruden \& Varnes (1996)}
\end{frame}
%%%%%%%%%%%%%%%%%%%%%%%%%%%%%%%%%%%%%%%%%%%%%%%%%%%%%%%%%%%%%%%%
\begin{frame}
\frametitle{Cruden \& Varnes (1996)}
\begin{figure}
\centering
\includegraphics[scale=0.7]{menm-glosary} 
\end{figure}
\tiny{Fuente: Cruden \& Varnes (1996)}
\end{frame}
%%%%%%%%%%%%%%%%%%%%%%%%%%%%%%%%%%%%%%%%%%%%%%%%%%%%%%%%%%%%%%%%
\begin{frame}
\frametitle{Actividad}
\framesubtitle{Estado}
\begin{figure}
\centering
\includegraphics[scale=0.52]{menm-actividad} 
\end{figure}
\tiny{Fuente: Cruden \& Varnes (1996)}
\end{frame}
%%%%%%%%%%%%%%%%%%%%%%%%%%%%%%%%%%%%%%%%%%%%%%%%%%%%%%%%%%%%%%%%
\begin{frame}
\frametitle{Actividad}
\framesubtitle{Estado}
\begin{figure}
\centering
\includegraphics[scale=0.7]{menm-actividad1} 
\end{figure}
\tiny{Fuente: Cruden \& Varnes (1996)}
\end{frame}
%%%%%%%%%%%%%%%%%%%%%%%%%%%%%%%%%%%%%%%%%%%%%%%%%%%%%%%%%%%%%%%%
\begin{frame}
\frametitle{Actividad}
\framesubtitle{Distribución}
\begin{figure}
\centering
\includegraphics[scale=0.48]{menm-distribucion} 
\end{figure}
\tiny{Fuente: Cruden \& Varnes (1996)}
\end{frame}
%%%%%%%%%%%%%%%%%%%%%%%%%%%%%%%%%%%%%%%%%%%%%%%%%%%%%%%%%%%%%%%%
\begin{frame}
\frametitle{Actividad}
\framesubtitle{Estilo}
\begin{figure}
\centering
\includegraphics[scale=0.48]{menm-estilo} 
\end{figure}
\tiny{Fuente: Cruden \& Varnes (1996)}
\end{frame}
%%%%%%%%%%%%%%%%%%%%%%%%%%%%%%%%%%%%%%%%%%%%%%%%%%%%%%%%%%%%%%%%
\begin{frame}
\frametitle{Velocidad}
\begin{figure}
\centering
\includegraphics[scale=0.65]{menm-velocidad} 
\end{figure}
\tiny{Fuente: Cruden \& Varnes (1996)}
\end{frame}
%%%%%%%%%%%%%%%%%%%%%%%%%%%%%%%%%%%%%%%%%%%%%%%%%%%%%%%%%%%%%%%%
\begin{frame}
\frametitle{Intensidad vs Magnitud}
\scriptsize{
\textbf{Intensidad del deslizamiento}: un grupo de parámetros distribuidos espacialmente relacionados con el poder destructivo del deslizamiento. Los parámetros pueden ser descritos cuantitativamente o cualitativamente y pueden incluir:
\vfill
\begin{itemize}
\item La velocidad máxima del movimiento,
\item El desplazamiento total,
\item El desplazamiento diferencial,
\item La profundidad de la masa desplazada,
\item El pico de descarga por unidad de ancho, la energía cinética por unidad de área.
\end{itemize}
\vfill
\textbf{Magnitud del deslizamiento}: la medida del tamaño del deslizamiento. Este puede ser cuantitativo o puede ser descrito por su volumen o indirectamente por su área. Los descriptores pueden referirse al escarpe del deslizamiento, al depósito del deslizamiento o ambos.
}
\tiny{Fuente: AGS (2007), Corominas et al. (2014)}
\end{frame}
%%%%%%%%%%%%%%%%%%%%%%%%%%%%%%%%%%%%%%%%%%%%%%%%%%%%%%%%%%%%%%%%
\begin{frame}
\frametitle{Propagación lateral}
\framesubtitle{\emph{Lateral spreading}}
\begin{figure}
\centering
\includegraphics[scale=0.62]{menm-propagacion} 
\end{figure}
\tiny{Fuente: Cruden \& Varnes (1996)}
\end{frame}
%%%%%%%%%%%%%%%%%%%%%%%%%%%%%%%%%%%%%%%%%%%%%%%%%%%%%%%%%%%%%%%%
\begin{frame}
\frametitle{Volcamiento}
\framesubtitle{\emph{Toppling}}
\begin{figure}
\centering
\includegraphics[scale=0.75]{menm-volcamiento} 
\end{figure}
\tiny{Fuente: Cruden \& Varnes (1996)}
\end{frame}
%%%%%%%%%%%%%%%%%%%%%%%%%%%%%%%%%%%%%%%%%%%%%%%%%%%%%%%%%%%%%%%%
\begin{frame}
\frametitle{Volcamiento}
\framesubtitle{\emph{Toppling}}
\begin{figure}
\centering
\includegraphics[scale=0.55]{menm-volcamiento1} 
\end{figure}
\end{frame}
%%%%%%%%%%%%%%%%%%%%%%%%%%%%%%%%%%%%%%%%%%%%%%%%%%%%%%%%%%%%%%%%
\begin{frame}
\frametitle{Caida}
\framesubtitle{\emph{Fall}}
\begin{figure}
\centering
\includegraphics[scale=0.55]{menm-caida} 
\end{figure}
\end{frame}
%%%%%%%%%%%%%%%%%%%%%%%%%%%%%%%%%%%%%%%%%%%%%%%%%%%%%%%%%%%%%%%%
\begin{frame}
\frametitle{Caída}
\framesubtitle{\emph{Fall}}
\begin{figure}
\centering
\includegraphics[scale=0.52]{menm-caida1} 
\end{figure}
\end{frame}
%%%%%%%%%%%%%%%%%%%%%%%%%%%%%%%%%%%%%%%%%%%%%%%%%%%%%%%%%%%%%%%%
\begin{frame}
\frametitle{Caída}
\framesubtitle{Depósitos de vertiente: Talus}
\begin{figure}
\centering
\includegraphics[scale=0.52]{talus} 
\end{figure}
\end{frame}
%%%%%%%%%%%%%%%%%%%%%%%%%%%%%%%%%%%%%%%%%%%%%%%%%%%%%%%%%%%%%%%%
\begin{frame}
\frametitle{Deslizamiento}
\framesubtitle{Planar \& Rotacional}
\begin{columns}
\begin{column}{0.5\textwidth}
\begin{figure}
\centering
\includegraphics[scale=0.5]{menm-planar} 
\end{figure}
\end{column}
\begin{column}{0.5\textwidth}
\begin{figure}
\centering
\includegraphics[scale=0.5]{menm-rotacional} 
\end{figure}
\end{column}
\end{columns}
\end{frame}
%%%%%%%%%%%%%%%%%%%%%%%%%%%%%%%%%%%%%%%%%%%%%%%%%%%%%%%%%%%%%%%%
\begin{frame}
\frametitle{Deslizamientos rotacionales}
\framesubtitle{\emph{Slides}}
\begin{figure}
\centering
\includegraphics[scale=0.52]{menm-rotacional1} 
\end{figure}
\end{frame}
%%%%%%%%%%%%%%%%%%%%%%%%%%%%%%%%%%%%%%%%%%%%%%%%%%%%%%%%%%%%%%%%
\begin{frame}
\frametitle{Deslizamientos rotacionales}
\framesubtitle{\emph{Slides}}
\begin{figure}
\centering
\includegraphics[scale=0.52]{menm-rotacional2} 
\end{figure}
\end{frame}
%%%%%%%%%%%%%%%%%%%%%%%%%%%%%%%%%%%%%%%%%%%%%%%%%%%%%%%%%%%%%%%%
\begin{frame}
\frametitle{Deslizamientos planares}
\framesubtitle{\emph{Slides}}
\begin{figure}
\centering
\includegraphics[scale=0.52]{menm-planar1} 
\end{figure}
\end{frame}
%%%%%%%%%%%%%%%%%%%%%%%%%%%%%%%%%%%%%%%%%%%%%%%%%%%%%%%%%%%%%%%%
\begin{frame}
\frametitle{Flujos}
\framesubtitle{flows}
\begin{figure}
\centering
\includegraphics[scale=0.5]{flujos4} 
\end{figure}
\end{frame}
%%%%%%%%%%%%%%%%%%%%%%%%%%%%%%%%%%%%%%%%%%%%%%%%%%%%%%%%%%%%%%%%
\begin{frame}
\frametitle{Flujos canalizados vs Flujos de ladera}
\begin{figure}
\centering
\includegraphics[scale=0.8]{flujos5} 
\end{figure}
\tiny{Fuente: Nettleton et al. (2005) }
\end{frame}
%%%%%%%%%%%%%%%%%%%%%%%%%%%%%%%%%%%%%%%%%%%%%%%%%%%%%%%%%%%%%%%%
\begin{frame}
\frametitle{Flujos}
\begin{figure}
\centering
\includegraphics[scale=0.6]{flujo8} 
\end{figure}
\end{frame}
%%%%%%%%%%%%%%%%%%%%%%%%%%%%%%%%%%%%%%%%%%%%%%%%%%%%%%%%%%%%%%%%
\begin{frame}
\frametitle{Flujos}
\begin{figure}
\centering
\includegraphics[scale=0.50]{flujo5} 
\end{figure}
\end{frame}
%%%%%%%%%%%%%%%%%%%%%%%%%%%%%%%%%%%%%%%%%%%%%%%%%%%%%%%%%%%%%%%%
\begin{frame}
\frametitle{Flujos canalizados}
\begin{figure}
\centering
\includegraphics[scale=0.5]{flujos6} 
\end{figure}
\end{frame}
%%%%%%%%%%%%%%%%%%%%%%%%%%%%%%%%%%%%%%%%%%%%%%%%%%%%%%%%%%%%%%%%
\begin{frame}
\frametitle{Reptación}
\framesubtitle{\emph{Creeping}}
\begin{figure}
\centering
\includegraphics[scale=0.80]{reptacion1} 
\end{figure}
\end{frame}
%%%%%%%%%%%%%%%%%%%%%%%%%%%%%%%%%%%%%%%%%%%%%%%%%%%%%%%%%%%%%%%%
\begin{frame}
\frametitle{Reptación}
\scriptsize{
La reptación de suelos ha sido tradicionalmente considerada como un conjunto de movimientos en masa lentos (continuos y estacionales), causado por el clima (temperatura y condiciones de humedad) y biota, y balanceado por la topografía, procesos de meteorización de la roca y tasa de producción del suelo.
\vfill
La reptación de suelo (creeping of the surface soil, Davis (1982)) fue inicialmente descrito como el resultado de la gravedad, fluctuaciones de la temperatura y la acción de la biota (crecimiento y decaimiento de raíces, actividad de gusanos, hormigas y otros fauna del suelo), retrabajado en un fenómeno permanente de dilatación y contracción en climas tropicales. 
\vfill
Davison (1889) propuso que durante el congelamiento y cabio del nivel freático: (i) expansión del suelo normal a la superficie, pero (ii) contracción vertical, y (iii) la cohesión del suelo previene desplazamientos paralelos a la superficie durante la expansión
\vfill
Movimiento tipo difusivo (diffusion-like) movimiento aleatorio de las partículas del suelo resultando en al dispersión desde regiones de alta concentración (densidad) a regiones de baja concentración (Kirkby, 1967).
\vfill
\centering
Los \textbf{flujos} implican movimientos mucho mas rápidos que la reptación, es decir pueden ser \textbf{perceptibles}, mientras que la \textbf{reptación} definitivamente es \textbf{imperceptible}.
}
\tiny{Fuente: Pawlik \& Samonil, 2018)}
\end{frame}
%%%%%%%%%%%%%%%%%%%%%%%%%%%%%%%%%%%%%%%%%%%%%%%%%%%%%%%%%%%%%%%%
\begin{frame}
\frametitle{Reptación}
\framesubtitle{Erosión}
\begin{figure}
\centering
\includegraphics[scale=0.85]{reptacion} 
\end{figure}
\end{frame}
%%%%%%%%%%%%%%%%%%%%%%%%%%%%%%%%%%%%%%%%%%%%%%%%%%%%%%%%%%%%%%%%
\end{document}